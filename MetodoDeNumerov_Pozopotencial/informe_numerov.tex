%% ============================================================
%%  Informe Técnico: Método de Numerov — Pozos Potenciales
%%  Compilar en Overleaf con pdfLaTeX
%%  Archivos requeridos junto al .tex:
%%    fig_caja.png, fig_pozo_V.png, fig_doble.png, fig_comparacion.png
%%    MetodoNumerov.f90
%% ============================================================
\documentclass[12pt,a4paper]{article}

%% ── Paquetes ────────────────────────────────────────────────────────────────
\usepackage[utf8]{inputenc}
\usepackage[T1]{fontenc}
\usepackage[spanish,es-nodecimaldot]{babel}
\usepackage{amsmath,amssymb,amsthm}
\usepackage{physics}            % \dv, \pdv, \bra, \ket …
\usepackage{graphicx}
\usepackage{booktabs}           % tablas bonitas
\usepackage{array}
\usepackage{xcolor}
\usepackage[colorlinks=true,
            linkcolor=blue!60!black,
            citecolor=green!50!black,
            urlcolor=blue!70!black]{hyperref}
\usepackage{geometry}
\usepackage{listings}
\usepackage{caption}
\usepackage{subcaption}
\usepackage{float}
\usepackage{siunitx}
\usepackage{enumitem}
\usepackage{microtype}
\usepackage{parskip}

\geometry{margin=2.5cm, top=2.5cm, bottom=3cm}

%% ── Colores personalizados ──────────────────────────────────────────────────
\definecolor{codebg}{HTML}{F5F5F5}
\definecolor{codeframe}{HTML}{CCCCCC}
\definecolor{kwcolor}{HTML}{0033B3}
\definecolor{strcolor}{HTML}{067D17}
\definecolor{cmtcolor}{HTML}{8C8C8C}

%% ── Estilo para código Fortran ──────────────────────────────────────────────
\lstdefinelanguage{Fortran90}{
  morekeywords={program, end, implicit, none, integer, double, precision,
                parameter, function, contains, result, select, case, do,
                while, if, then, else, open, close, write, print, read,
                real, logical},
  sensitive=false,
  morecomment=[l]{!},
  morestring=[b]",
}
\lstset{
  language=Fortran90,
  backgroundcolor=\color{codebg},
  frame=single, framerule=0.6pt, rulecolor=\color{codeframe},
  basicstyle=\ttfamily\footnotesize,
  keywordstyle=\color{kwcolor}\bfseries,
  stringstyle=\color{strcolor},
  commentstyle=\color{cmtcolor}\itshape,
  numbers=left, numberstyle=\tiny\color{gray}, numbersep=6pt,
  breaklines=true, breakatwhitespace=true,
  tabsize=4, captionpos=b,
  showstringspaces=false,
}

%% ── Comandos útiles ─────────────────────────────────────────────────────────
\newcommand{\eo}{E_0}
\newcommand{\kk}{k^2}

%% ─────────────────────────────────────────────────────────────────────────────
\begin{document}

%% ── Portada ─────────────────────────────────────────────────────────────────
\begin{titlepage}
  \centering
  \vspace*{1cm}
  {\large\bfseries Universidad — Física Computacional\par}
  \vspace{0.4cm}
  {\small Modelamiento en Física Computacional\par}
  \vspace{2.5cm}
  {\LARGE\bfseries Método de Numerov para la Determinación\\[6pt]
   de Niveles de Energía en Pozos Potenciales\par}
  \vspace{0.8cm}
  {\large Informe Técnico\par}
  \vspace{3cm}
  \rule{0.5\linewidth}{0.5pt}\\[0.4cm]
  {\large\itshape Implementación en Fortran 90\par}
  \rule{0.5\linewidth}{0.5pt}
  \vfill
  {\large Febrero 2026\par}
\end{titlepage}

%% ── Índice ──────────────────────────────────────────────────────────────────
\tableofcontents
\newpage

%% ─────────────────────────────────────────────────────────────────────────────
\section{Introducción}
\label{sec:intro}

La ecuación de Schrödinger independiente del tiempo (TISE, por sus siglas en
inglés) constituye el fundamento de la mecánica cuántica no relativista.  Su
resolución analítica exacta sólo es posible para unos pocos potenciales modelo;
en todos los demás casos se requieren técnicas numéricas.

El \textbf{método de Numerov} \cite{numerov1924} es un algoritmo de integración
de segundo orden diseñado específicamente para ecuaciones diferenciales
ordinarias (EDO) de la forma
\begin{equation}\label{eq:generic_ode}
  \psi''(x) = f(x)\,\psi(x),
\end{equation}
donde no aparece la derivada de primer orden.  Al aplicar la TISE unidimensional
en unidades naturales ($\hbar = 1$, $m = 1$), ésta adopta exactamente esa forma,
lo que convierte al método de Numerov en la herramienta ideal para calcular
estados ligados cuánticos.

En este trabajo se estudian tres pozos potenciales representativos:
\begin{itemize}[leftmargin=*]
  \item \textbf{Pozo cuadrado finito}: modelo fundamental con solución
        analítica exacta mediante ecuaciones trascendentes.
  \item \textbf{Pozo en V (lineal)}: solución analítica mediante funciones
        de Airy, lo que permite validar con gran precisión el método numérico.
  \item \textbf{Doble pozo cuártico}: sistema sin solución de forma cerrada,
        con rica fenomenología cuántica (desdoblamiento por efecto túnel).
\end{itemize}

La implementación se realizó en Fortran~90 y combina el propagador de Numerov
con un \emph{shooting method} (método de disparo) y búsqueda de raíces por
bisección.

%% ─────────────────────────────────────────────────────────────────────────────
\section{Marco Teórico}
\label{sec:teoria}

\subsection{Ecuación de Schrödinger independiente del tiempo}
\label{ssec:tise}

En unidades naturales $\hbar = 1$ con $m = 1$, la TISE unidimensional es
\begin{equation}\label{eq:tise}
  -\frac{1}{2}\,\psi''(x) + V(x)\,\psi(x) = E\,\psi(x),
\end{equation}
que al despejar $\psi''$ toma la forma canónica de Numerov:
\begin{equation}\label{eq:numerov_form}
  \psi''(x) = -k^2(x)\,\psi(x),
  \qquad k^2(x) \equiv 2\bigl[E - V(x)\bigr].
\end{equation}

La función $k^2(x)$ es positiva en las regiones clásicamente permitidas ($E >
V$) y negativa en las regiones prohibidas ($E < V$).  Las soluciones ligadas
satisfacen $\psi(x) \to 0$ cuando $|x| \to \infty$.

\subsection{Derivación del método de Numerov}
\label{ssec:numerov}

Considerando una malla uniforme $x_j = x_0 + j h$ con paso $h$, las
expansiones de Taylor de $\psi$ en $x_{j\pm 1}$ son:
\begin{align}
  \psi_{j+1} &= \psi_j + h\psi_j' + \frac{h^2}{2}\psi_j''
               + \frac{h^3}{6}\psi_j'''
               + \frac{h^4}{24}\psi_j^{(4)} + O(h^5),\\
  \psi_{j-1} &= \psi_j - h\psi_j' + \frac{h^2}{2}\psi_j''
               - \frac{h^3}{6}\psi_j'''
               + \frac{h^4}{24}\psi_j^{(4)} + O(h^5).
\end{align}

Sumando ambas expresiones y eliminando las potencias impares:
\begin{equation}\label{eq:taylor_sum}
  \psi_{j+1} + \psi_{j-1} = 2\psi_j + h^2\psi_j'' + \frac{h^4}{12}\psi_j^{(4)}
                             + O(h^6).
\end{equation}

La clave del método de Numerov consiste en aproximar la derivada de cuarto
orden usando la derivada segunda en puntos vecinos.  Como $\psi'' = -k^2\psi$,
se tiene:
\begin{equation}\label{eq:fourth_der}
  \psi_j^{(4)} \approx
  \frac{-k^2_{j+1}\psi_{j+1} + 2\,k^2_j\,\psi_j - k^2_{j-1}\psi_{j-1}}{h^2}
  + O(h^2).
\end{equation}

Sustituyendo \eqref{eq:fourth_der} en \eqref{eq:taylor_sum} y reordenando,
se obtiene la \textbf{fórmula de recurrencia de Numerov}:
\begin{equation}\label{eq:numerov_recurr}
  \boxed{
  \psi_{j+1} = \frac{2\!\left(1 - \tfrac{5h^2}{12}\,k^2_j\right)\psi_j
                    - \left(1 + \tfrac{h^2}{12}\,k^2_{j-1}\right)\psi_{j-1}}
                    {1 + \tfrac{h^2}{12}\,k^2_{j+1}}
  }
\end{equation}

El método es de orden 4 en $h$; el error de truncamiento local es
$O(h^6/\psi)$, lo que lo hace significativamente más preciso que el método de
Verlet estándar para este tipo de ecuaciones.

\paragraph{Ventajas frente a otros integradores.}
A diferencia de los métodos de Runge-Kutta aplicados al sistema de dos
ecuaciones de primer orden, Numerov explota la ausencia del término $\psi'$ y
logra orden 4 con sólo tres evaluaciones por paso, en lugar de cuatro.

\subsection{Método de disparo (\emph{shooting method})}
\label{ssec:shooting}

Para resolver el problema de valores de frontera $\psi(\pm L) \approx 0$ (donde
$L = 4$ u.a.\ en la implementación), se utiliza el \emph{método de disparo
hacia la derecha}:

\begin{enumerate}[label=\arabic*.]
  \item Fijar condiciones iniciales en el borde izquierdo:
        $\psi_0 = 0$, $\psi_1 = \varepsilon$ (con $\varepsilon = 10^{-5}$).
  \item Propagar con la recurrencia \eqref{eq:numerov_recurr} hasta
        el borde derecho y registrar el valor final $\psi_n$.
  \item Repetir para distintos valores de $E$ y buscar los cambios de signo
        de $\psi_n(E)$.  Un cambio de signo indica la presencia de un eigenvalor
        en ese intervalo de energía.
\end{enumerate}

La condición de frontera $\psi(x_L) = 0$ es una buena aproximación cuando la
función de onda decae suficientemente rápido en la región prohibida, lo que
requiere $\kappa \cdot d \gg 1$, donde $\kappa = \sqrt{2(V_0-E)}$ y $d$ es la
distancia desde el borde del pozo hasta la frontera del dominio.

\subsection{Refinamiento por bisección}
\label{ssec:biseccion}

Una vez detectado un cambio de signo en el intervalo $[E_1, E_2]$, el eigenvalor
se afina mediante el \textbf{método de bisección}: en cada iteración se evalúa
$\psi_n(E_m)$ con $E_m = (E_1+E_2)/2$ y se actualiza el intervalo conservando
el cambio de signo.  Con 40 iteraciones el intervalo inicial de amplitud
$\Delta E = 0.2$ u.a.\ se reduce a $\Delta E/2^{40} \approx 1.8 \times
10^{-13}$ u.a., garantizando convergencia en doble precisión.

%% ─────────────────────────────────────────────────────────────────────────────
\section{Pozos Potenciales Considerados}
\label{sec:pozos}

\subsection{Pozo cuadrado finito}

\begin{equation}\label{eq:V_caja}
  V_1(x) =
  \begin{cases}
    0    & \text{si } |x| < a = 1.5 \text{ u.a.},\\
    V_0 = 25 & \text{en otro caso.}
  \end{cases}
\end{equation}

Las energías de enlace satisfacen las ecuaciones trascendentes (estados pares
e impares respectivamente):
\begin{equation}\label{eq:caja_analitica}
  k\tan(ka) = \kappa, \qquad -k\cot(ka) = \kappa,
\end{equation}
con $k = \sqrt{2E}$ y $\kappa = \sqrt{2(V_0 - E)}$.

\subsection{Pozo en V (potencial lineal)}

\begin{equation}\label{eq:V_lineal}
  V_2(x) = F|x|, \qquad F = 4 \text{ u.a.}
\end{equation}

La TISE con este potencial se reduce, mediante el cambio de variables
$z = F^{1/3}(x - E/F) \cdot 2^{1/3}$, a la ecuación de Airy:
\begin{equation}\label{eq:airy}
  \frac{d^2 \psi}{d z^2} = z\,\psi.
\end{equation}

Para un potencial simétrico la condición de paridad en $x = 0$ selecciona:
\begin{itemize}
  \item \textbf{Estados pares:} $\operatorname{Ai}'(-E/2) = 0$
        $\Rightarrow E_n = 2\,z_n^{(\text{par})}$
  \item \textbf{Estados impares:} $\operatorname{Ai}(-E/2) = 0$
        $\Rightarrow E_n = 2\,z_n^{(\text{imp})}$
\end{itemize}
donde $z_n^{(\text{par})}$ y $z_n^{(\text{imp})}$ son los ceros de $\text{Ai}'$
y $\text{Ai}$ en el eje negativo, respectivamente.

\subsection{Doble pozo cuártico}

\begin{equation}\label{eq:V_doble}
  V_3(x) = \frac{\alpha}{2}\,x^4 - \beta\,x^2,
  \qquad \alpha = 1, \quad \beta = 5 \text{ u.a.}
\end{equation}

Los mínimos del potencial se encuentran en $x_{\min} = \pm\sqrt{\beta/\alpha}
= \pm\sqrt{5} \approx \pm 2.236$ u.a., con $V_{\min} = -\beta^2/(2\alpha) =
-12.5$ u.a.  La barrera central está en $x = 0$, con $V(0) = 0$.  Este potencial
posee la rica fenomenología del \textbf{efecto túnel cuántico}: los estados de
menor energía presentan un desdoblamiento en pares casi degenerados
(''splitting'' túnel), cuya separación es exponencialmente pequeña para estados
profundos.

%% ─────────────────────────────────────────────────────────────────────────────
\section{Implementación en Fortran~90}
\label{sec:implementacion}

El código \texttt{MetodoNumerov.f90} implementa en un único programa modular
los tres componentes descritos: propagador de Numerov, barrido en energía y
refinamiento por bisección.  La arquitectura se controla con la variable entera
\texttt{tipo\_pozo} $\in \{1, 2, 3\}$.

\vspace{4pt}
\begin{lstlisting}[caption={Función de potencial y propagador de Numerov (fragmento).},
                   label={lst:numerov}]
! --- Potencial V(x) ---
function V_func(pos, selector)
    double precision :: pos, V_func
    integer :: selector
    select case (selector)
    case(1) ! Caja Cuadrada
        if (abs(pos) < 1.5d0) then
            V_func = 0.d0
        else
            V_func = 25.d0
        end if
    case(2) ! Pozo en V (Lineal)
        V_func = 4.0d0 * abs(pos)
    case(3) ! Doble Pozo cuartico
        V_func = 0.5d0*pos**4 - 5.0d0*pos**2
    end select
end function

! --- Propagador de Numerov ---
function solve_numerov(E, n, h, x, selector) result(psif)
    double precision :: E, h, x(0:n), psif, p(0:n), k2(0:n), h2_12
    integer :: n, j, selector
    h2_12 = (h**2)/12.d0
    do j = 0, n
        k2(j) = 2.0d0 * (E - V_func(x(j), selector))
    end do
    p(0) = 0.d0;  p(1) = 1e-5
    do j = 1, n-1
        p(j+1) = (2.d0*(1.d0-5.d0*h2_12*k2(j))*p(j)         &
                 - (1.d0+h2_12*k2(j-1))*p(j-1))               &
                 / (1.d0+h2_12*k2(j+1))
        if (abs(p(j+1)) > 1e10) p(j+1) = 1e10 ! control de desborde
    end do
    psif = p(n)
end function
\end{lstlisting}

La malla se define con $N = 2000$ puntos en $x \in [-4, 4]$ u.a., dando un
paso $h = 8/2000 = 0.004$ u.a.  El error de truncamiento de Numerov es $O(h^4)
\approx (0.004)^4 = 2.56 \times 10^{-10}$, suficiente para la doble precisión
de Fortran.

El barrido en energía recorre el intervalo $E \in [-15, 20]$ u.a.\ con paso
$\Delta E = 0.2$ u.a.\ y la bisección refina cada eigenvalor con 40
iteraciones.

%% ─────────────────────────────────────────────────────────────────────────────
\section{Resultados y Análisis}
\label{sec:resultados}

\subsection{Pozo cuadrado finito}
\label{ssec:res_caja}

\begin{figure}[H]
  \centering
  \includegraphics[width=0.78\textwidth]{fig_caja.png}
  \caption{Pozo cuadrado finito ($V_0 = 25$ u.a., $a = 1.5$ u.a.).  Se muestran
           los seis niveles energéticos reportados por el código junto con las
           funciones de onda correspondientes (escaladas para visualización).
           Los niveles en rojo más claro corresponden a artefactos numéricos
           generados por la gestión del desbordamiento (véase §\ref{ssec:overflow}).}
  \label{fig:caja}
\end{figure}

La tabla~\ref{tab:caja} compara los niveles encontrados por el código con los
valores exactos obtenidos resolviendo las ecuaciones trascendentes
\eqref{eq:caja_analitica} con bisección de alta precisión ($\Delta E =
10^{-3}$, con preservación correcta del signo).

\begin{table}[H]
  \centering
  \caption{Comparación de niveles de energía del pozo cuadrado finito (u.a.).
           Los niveles marcados con $\dagger$ son artefactos numéricos;
           los marcados con $\star$ son niveles físicos no detectados.}
  \label{tab:caja}
  \begin{tabular}{clccr}
    \toprule
    $n$ & Estado & $E_n$ exacto & $E_n$ código & $\delta E$ \\
    \midrule
    1 & $1s$ (par)   & $0.4588$  & ---           & {\footnotesize $\star$ no detectado} \\
    2 & $1p$ (impar) & $1.8321$  & ---           & {\footnotesize $\star$ no detectado} \\
    3 & $2s$ (par)   & $4.1101$  & $4.1101$      & $< 10^{-4}$ \\
    — & (artefacto)  & ---       & $4.5968^\dagger$ & {\footnotesize espurio} \\
    — & (artefacto)  & ---       & $5.3405^\dagger$ & {\footnotesize espurio} \\
    4 & $2p$ (impar) & $7.2735$  & $7.2735$      & $< 10^{-4}$ \\
    5 & $3s$ (par)   & $11.2865$ & $11.2865$     & $< 10^{-4}$ \\
    6 & $3p$ (impar) & $16.0767$ & $16.0767$     & $< 10^{-4}$ \\
    \bottomrule
  \end{tabular}
\end{table}

El código detecta correctamente los cuatro niveles superiores con errores
menores a $10^{-4}$ u.a., demostrando la alta precisión intrínseca del
propagador de Numerov.  Sin embargo, los dos niveles más bajos ($E_1 = 0.459$
y $E_2 = 1.832$ u.a.) no son detectados, y dos niveles espurios aparecen en
su lugar.  Esta discrepancia se analiza en la sección~\ref{ssec:overflow}.

\subsection{Pozo en V (potencial lineal)}
\label{ssec:res_V}

\begin{figure}[H]
  \centering
  \includegraphics[width=0.78\textwidth]{fig_pozo_V.png}
  \caption{Pozo en V con $V(x) = 4|x|$.  Se grafican los ocho primeros niveles
           energéticos y sus funciones de onda (escaladas).  El estado
           fundamental tiene energía $E_1 \approx 2.038$ u.a.}
  \label{fig:V}
\end{figure}

El pozo en~V es el único de los tres potenciales que dispone de solución analítica
exacta expresable en términos de ceros de las funciones de Airy, permitiendo
una validación rigurosa.

\begin{table}[H]
  \centering
  \caption{Niveles de energía del pozo en~V: comparación entre el resultado
           numérico (Numerov, Fortran) y el analítico (ceros de funciones Airy).}
  \label{tab:V}
  \begin{tabular}{clccc}
    \toprule
    $n$ & Paridad & $E_n$ Airy (exacto) & $E_n$ Numerov & Error relativo \\
    \midrule
    1 & par  & 2.03758 & 2.03758 & $< 10^{-5}$ \\
    2 & impar & 4.67621 & 4.67621 & $< 10^{-5}$ \\
    3 & par  & 6.49640 & 6.49639 & $1.5\times10^{-6}$ \\
    4 & impar & 8.17590 & 8.17591 & $1.2\times10^{-6}$ \\
    5 & par  & 9.64020 & 9.64042 & $2.3\times10^{-5}$ \\
    6 & impar & 11.04112 & 11.04333 & $2.0\times10^{-4}$ \\
    7 & par  & 12.32661 & 12.34012 & $1.1\times10^{-3}$ \\
    8 & impar & 13.57342 & 13.63253 & $4.3\times10^{-3}$ \\
    \bottomrule
  \end{tabular}
\end{table}

La concordancia entre los valores numéricos y los analíticos es notable: los
cuatro primeros niveles coinciden con errores inferiores a $10^{-5}$, y el error
crece suavemente para niveles superiores.  Este incremento es esperable: los
estados de mayor energía tienen funciones de onda con mayor frecuencia espacial
y se acercan al borde del dominio $|x| = 4$~u.a., donde el potencial lineal
$V(4) = 16$~u.a.\ no confina perfectamente.

\paragraph{Validación del método.}
Este resultado valida experimentalmente el orden de precisión del método de
Numerov: el error relativo para los niveles bajos ($\sim 10^{-5}$) es coherente
con el error de truncamiento teórico $O(h^4) = O((0.004)^4) \approx 2.6\times
10^{-10}$ magnificado por la acumulación a lo largo de los $N = 2000$ pasos,
resultando en un error global de orden $O(h^4 N) \approx O(h^3) \approx
6\times10^{-8}$ por nivel, más efectos de frontera.

\subsection{Doble pozo cuártico}
\label{ssec:res_doble}

\begin{figure}[H]
  \centering
  \includegraphics[width=0.78\textwidth]{fig_doble.png}
  \caption{Doble pozo cuártico $V(x) = \tfrac{1}{2}x^4 - 5x^2$.  La línea
           punteada roja indica el tope de la barrera central ($E = 0$); la
           azul señala el mínimo del pozo ($E = -12.5$ u.a.).  Los ocho
           niveles reportados están todos por encima del tope de la barrera.}
  \label{fig:doble}
\end{figure}

\begin{table}[H]
  \centering
  \caption{Niveles de energía del doble pozo cuártico encontrados por el código
           (todos ellos por encima del tope de la barrera $E = 0$).}
  \label{tab:doble}
  \begin{tabular}{cl}
    \toprule
    $n$ & $E_n$ (u.a.) \\
    \midrule
    1 & $+0.03225$ \\
    2 & $+0.90371$ \\
    3 & $+2.73805$ \\
    4 & $+4.55182$ \\
    5 & $+6.60208$ \\
    6 & $+8.79476$ \\
    7 & $+11.12476$ \\
    8 & $+13.57853$ \\
    \bottomrule
  \end{tabular}
\end{table}

Todos los niveles hallados satisfacen $E > 0$, es decir, se encuentran por
encima del tope de la barrera central.  Para estos estados, la región
clásicamente permitida es \emph{conexa} (un solo intervalo continuo alrededor
de $x = 0$), por lo que la física es la de un pozo único con forma cuártica.

La fenomenología más interesante del doble pozo —el \textbf{desdoblamiento
túnel} de los estados bajo la barrera ($-12.5 < E < 0$)— no es accesible con
el código en su estado actual (véase §\ref{ssec:overflow}).  En el límite de
barrera alta ($\beta \gg \alpha$), la energía de desdoblamiento es
\begin{equation}\label{eq:tunneling}
  \Delta E_n \approx \frac{\omega}{\pi}\,
  \exp\!\left(-\int_{-x_{\text{tp}}}^{x_{\text{tp}}} \kappa(x)\,dx\right),
\end{equation}
donde $\kappa(x) = \sqrt{2(V(x) - E)}$ y $\omega$ es la frecuencia de pequeñas
oscilaciones en el fondo del pozo.

Para estimar $\omega$ se expande el potencial alrededor del mínimo:
\begin{equation}
  V(x) \approx V_{\min} + \frac{1}{2}\,V''(x_{\min})\,(x-x_{\min})^2,
\end{equation}
con $V''(x) = 6x^2 - 10$, de modo que $V''(\sqrt{5}) = 20$, y por tanto
$\omega = \sqrt{V''/m} = \sqrt{20} \approx 4.47$ u.a.  La energía del estado
fundamental dentro de cada pozo es $E_0^{\text{loc}} \approx V_{\min} + \omega/2
= -12.5 + 2.24 = -10.26$~u.a.

\subsection{Comparación global}

La figura~\ref{fig:comp} muestra los tres potenciales con sus niveles en un
único panel comparativo.

\begin{figure}[H]
  \centering
  \includegraphics[width=\textwidth]{fig_comparacion.png}
  \caption{Panel comparativo de los tres pozos potenciales.  De izquierda a
           derecha: pozo cuadrado, pozo en~V y doble pozo cuártico.  Los
           niveles de energía se representan como líneas horizontales sobre cada
           potencial.}
  \label{fig:comp}
\end{figure}

%% ─────────────────────────────────────────────────────────────────────────────
\section{Análisis de Limitaciones Numéricas}
\label{sec:limitaciones}

\subsection{El problema del desbordamiento y la preservación del signo}
\label{ssec:overflow}

El propagador de Numerov puede producir valores de $\psi_j$ que crecen
exponencialmente en las regiones clásicamente prohibidas.  En la implementación,
la línea de control de desbordamiento es:

\begin{lstlisting}[numbers=none]
    if (abs(p(j+1)) > 1e10) p(j+1) = 1e10
\end{lstlisting}

Esta instrucción siempre asigna el valor \emph{positivo} $+10^{10}$, perdiendo
la información de signo.  El correcto tratamiento debería ser:

\begin{lstlisting}[numbers=none]
    if (abs(p(j+1)) > 1e10) p(j+1) = sign(1.d0, p(j+1)) * 1e10
\end{lstlisting}

Las consecuencias de esta omisión son:

\begin{enumerate}
  \item \textbf{Niveles perdidos.}  Para estados de baja energía (profundamente
        ligados), la función de onda en la región de la barrera derecha crece
        exponencialmente y excede $10^{10}$ con valor \emph{negativo} (que
        indica que el eigenvalor acaba de cruzarse).  Al recortarse a $+10^{10}$,
        el cambio de signo queda enmascarado: el barrido de energía no detecta
        la raíz.  Así, para el pozo cuadrado se pierden $E_1 = 0.459$ y $E_2
        = 1.832$~u.a.

  \item \textbf{Niveles espurios.}  En el intervalo de energía donde el
        desbordamiento recortado alterna entre $-|\psi_n|$ (valor real
        negativo) y $+10^{10}$ (recorte incorrecto), el algoritmo detecta
        transiciones de signo ficticias.  Estos artefactos generan los niveles
        espurios $E = 4.597$ y $E = 5.340$~u.a.\ en el pozo cuadrado.

  \item \textbf{Niveles de doble pozo por debajo de la barrera.}  La barrera
        central del doble pozo cuártico tiene una amplitud espacial
        $\sim 2\sqrt{5} \approx 4.47$ u.a.\ y el coeficiente de amortiguamiento
        $\kappa(0) = \sqrt{2\,|E_{\min}|} \approx 5$~u.a.$^{-1}$ para los
        estados más profundos.  Esto produce amplificaciones del orden
        $e^{\kappa \cdot d} \sim e^{22} \approx 3.6\times10^9$, fácilmente
        superiores al umbral de desbordamiento.
\end{enumerate}

\begin{table}[H]
  \centering
  \caption{Resumen del impacto del error de desbordamiento en cada pozo.}
  \label{tab:overflow}
  \begin{tabular}{lccl}
    \toprule
    Pozo & Niveles correctos & Niveles perdidos & Niveles espurios \\
    \midrule
    Caja cuadrada & 4 & 2 ($E_1$, $E_2$) & 2 ($E \approx 4.6$, 5.3) \\
    Pozo en V     & 8 & 0                & 0 \\
    Doble pozo    & 8 (sobre barrera) & $\gtrsim 8$ (bajo barrera) & posiblemente 0 \\
    \bottomrule
  \end{tabular}
\end{table}

El pozo en~V no sufre el problema porque la región clásicamente prohibida es
semi-infinita sin paredes reflectantes; la función de onda en el dominio
computacional es lo suficientemente pequeña en los bordes para no desbordar.

\subsection{Condición de frontera y dominio finito}

La condición de Dirichlet $\psi(\pm L) = 0$ es válida cuando la longitud de
decaimiento $1/\kappa$ es pequeña comparada con la distancia del borde del
pozo al extremo del dominio.  Para el pozo cuadrado con $V_0 = 25$ y
$E \approx 4$~u.a.:
\begin{equation}
  \kappa = \sqrt{2(V_0 - E)} = \sqrt{42} \approx 6.5 \text{ u.a.}^{-1},
  \qquad e^{-\kappa \cdot 2.5} \approx e^{-16.3} \approx 8\times10^{-8},
\end{equation}
lo que confirma que la aproximación es excelente para los niveles detectados.
Para los niveles más altos (cercanos a $V_0$), $\kappa \to 0$ y la aproximación
se deteriora; esto explica el aumento del error en los niveles 7 y~8 del pozo
en~V.

\subsection{Paso de barrido $\Delta E$ y resolución de raíces}

El paso $\Delta E = 0.2$ u.a.\ puede \emph{saltar} sobre un eigenvalor si éste
se encuentra en un intervalo donde la función $\psi_n(E)$ cambia de signo
dentro de una ventana muy estrecha.  Esto ocurre cuando la amplificación
exponencial es enorme: el cambio de signo en $\psi_n$ sucede en un rango de
energías $\delta E \sim h^2 / e^{\kappa L}$, que puede ser mucho menor que
$\Delta E$.  La solución correcta es usar $\Delta E$ más pequeño o bien rescalar
la función de onda a cada paso (técnica de Prüfer).

%% ─────────────────────────────────────────────────────────────────────────────
\section{Conclusiones}
\label{sec:conclusiones}

\begin{enumerate}[leftmargin=*, label=\arabic*.]
  \item El \textbf{método de Numerov} fue implementado exitosamente en
        Fortran~90 y combinado con un método de disparo y bisección para
        determinar eigenvalores del operador de Schrödinger en tres potenciales
        cualitativamente distintos.

  \item Para el \textbf{pozo en~V}, la comparación con las soluciones analíticas
        exactas (funciones de Airy) confirma que los errores numéricos son
        inferiores a $10^{-5}$ para los niveles bajos, creciendo hasta
        $\sim 4\times10^{-3}$ para el octavo nivel por efectos de borde.  Esto
        valida experimentalmente el orden $O(h^4)$ del integrador de Numerov.

  \item Para el \textbf{pozo cuadrado}, cuatro de los seis niveles físicos son
        detectados correctamente.  Los dos niveles de menor energía no son
        encontrados debido a un error en la gestión del desbordamiento numérico
        (la instrucción \texttt{p(j+1) = 1e10} debería ser
        \texttt{p(j+1) = sign(1.d0,p(j+1))*1e10}).

  \item Para el \textbf{doble pozo cuártico}, el código reporta correctamente
        los estados por encima de la barrera central ($E > 0$).  Los estados
        ligados en los pozos ($-12.5 < E < 0$), que exhiben el fenómeno de
        \emph{desdoblamiento túnel}, no son accesibles con el algoritmo actual.
        Se estima una energía de estado fundamental local de
        $E_0^{\text{loc}} \approx -10.3$~u.a.

  \item La corrección del manejo del signo en el control de desbordamiento
        permitiría al código encontrar todos los estados ligados de los tres
        pozos, incluidos los pares casi degenerados del doble pozo.

  \item El método de Numerov representa una elección óptima para este tipo de
        problemas: combina alta precisión (orden~4) con baja complejidad
        computacional (tres evaluaciones por paso), siendo superior a métodos
        genéricos de Runge-Kutta para la resolución de la TISE.
\end{enumerate}

%% ─────────────────────────────────────────────────────────────────────────────
\begin{thebibliography}{9}

\bibitem{numerov1924}
  B.~V.~Numerov,
  \textit{A method of extrapolation of perturbations},
  Monthly Notices of the Royal Astronomical Society, \textbf{84}, 592--601 (1924).

\bibitem{griffiths}
  D.~J.~Griffiths \& D.~F.~Schroeter,
  \textit{Introduction to Quantum Mechanics}, 3rd ed.
  Cambridge University Press (2018).

\bibitem{numrec}
  W.~H.~Press, S.~A.~Teukolsky, W.~T.~Vetterling, B.~P.~Flannery,
  \textit{Numerical Recipes in Fortran~90}, 2nd ed.
  Cambridge University Press (1996).

\bibitem{landau_lifshitz}
  L.~D.~Landau \& E.~M.~Lifshitz,
  \textit{Quantum Mechanics: Non-Relativistic Theory}, 3rd ed.
  Butterworth-Heinemann (1977).

\bibitem{abramowitz}
  M.~Abramowitz \& I.~A.~Stegun,
  \textit{Handbook of Mathematical Functions}.
  Dover Publications (1972).

\end{thebibliography}

\end{document}
